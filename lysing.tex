\documentclass[a4paper,11pt]{article}

\usepackage[utf8]{inputenc}
\usepackage[T1]{fontenc}
\usepackage{a4,graphics,amsmath,amsfonts,amsbsy,amssymb,amsthm}
\usepackage{graphicx}
\usepackage{hyperref}
\usepackage{float}
\usepackage{listings}
\usepackage{enumerate}
\usepackage{comment}
\usepackage{color}
\usepackage{accents}
\usepackage{epstopdf}

\usepackage[icelandic]{babel}
\setcounter{secnumdepth}{-1}

\title{Lýsing} \author{Matthías Páll Gissurarson  \and Sólrún Halla Einarsdóttir \and Ívar Örn Ragnarsson}

\begin{document}
\maketitle
\section{Rannsóknarvinna}
\subsection{Íþróttagrein}
Við höfum ákveðið að vinna með körfubolta, en það er helst
vegna þess að fyrir hann er hægt að finna góð gagnasöfn marga áratugi
aftur í tímann, margir leikir eru spilaðir í íþróttinni og mikið af
upplýsingum er hægt að fá um hvert lið og hvern leik. Einnig var
áhrifaþáttur að liðin eru oftast með töff nöfn.

\subsection{Spálíkön}
Hugmyndin er að hafa líkan þar sem hver vinnur og stigamunur sem unnin
er með spilar inn. Þá mundu öll liðin byrja jöfn, og svo er farið
eftir viðureignum og liðunum gefið auknar vigtir eftir því hvaða lið
þeir vinna, hve öflugt það lið er og með hve miklum mun þau unnu það
með. Þá er hugmyndin að láta öll lið byrja með ákveðið mikinn
styrkleika, og hækka svo og lækka styrkinn hjá þeim sem vinna og tapa
eftir því hve mikill stigamunur var á þeim og svo margfalda með
hlutfallinu á milli núverandi styrkleikja þeirra. Við eigum þó eftir
að prófa okkur áfram með þessi líkön.

\subsection{Tekjur}
Hægt væri að hafa tekjur af kerfinu með því að rukka fyrir áskrift af
því, en hún gæti verið tímabundin eða notkunarbundin. Einnig væri hægt
að hafa auglýsingar. Hugsanlegt er að hægt sé að setja upp veðmál,
þ.e. fólk veðjar gegn kerfinu eða gegn hvort öðru, og mundi kerfið þá
taka þeim veðmálum eða ekki, svipað og 1X2 gerir. Væru þá stuðlar
reiknaðir útfrá líkunum á því hvort liðið vinni.

\subsection{Önnur nýting}
Kerfið mætti nýta fyrir aðra hluti þar sem verið er að keppa,
t.d. eitthvað kerfi álíka Compare Hotness (en þá væru stig reiknuð út
frá ``viðureignum'').


\section{Notendasögur}
\begin{itemize}
\item Titill: Sækja gögn\\
  Lýsing: Sækja gögn um úrslit NBA-leikja af basketball-reference.com.\\
  Forgangur: 40\\
  Tími: 1 dagur\\
\item
  Titill: Geyma gögn í gagnagrunni\\
  Lýsing: Útbúa gagnagrunn sem heldur utan um gögn um úrslit leikja með skipulegum hætti svo þægilegt verði að vinna með þau.\\
  Forgangur: 40\\
  Tími: 2 dagar\\
\item
  Titill: Spá fyrir um úrslit\\
  Lýsing: Útbúa kerfi sem spáir fyrir um úrslit leiks milli gefinna liða í NBA\\
  Forgangur: 40\\
  Tími: 6 dagar\\
\item
  Titill: Vefþjónn\\
  Lýsing: Setjum upp vefþjón svo hægt sé að nota kerfið gegnum vafra\\
  Forgangur: 30\\
  Tími: 4 dagar\\
\item
  Titill: Aðgangskerfi\\
  Lýsing: Útbúa aðgangskerfi á vefþjón svo notendur geti skráð sig inn.\\
  Forgangur: 20\\
  Tími: 2 dagar\\
\item
  Titill: Halda utan um upplýsingar um notendur\\
  Lýsing: Útbúa kerfi sem heldur utan um notendanöfn, lykilorð og hversu mikið hver notandi hefur notað kerfið.\\
  Forgangur: 20\\
  Tími: 1 dagur\\
\item
  Titill: Nýta kerfið á öðrum vettvangi\\
  Lýsing: Prófa kerfið á öðrum deildum/íþróttagreinum.\\
  Forgangur: 10 \\
  Tími: 2 dagar\\
\item
  Titill: Áskrift\\
  Lýsing: Útbúa kerfi sem leyfir notendum að greiða fyrir notkun kerfis með ýmsum greiðsluleiðum. \\
  Forgangur: 10 \\
  Tími: 2 dagar\\
\item
  Titill: Rannsóknir\\
  Lýsing: Rannsaka hvað þarf í kerfið, hvað er hægt að nota til að útbúa það og hvaða gagnasöfn er hægt að hafa.\\
  Forgangur: 50\\
  Tími: 1 dagur\\
\item
  Titill: Skipulag\\
  Lýsing: Gerð notendasagna og verkáætlunnar.\\
  Forgangur: 50\\
  Tími: 1 dagur\\
\end{itemize}
\section{Verkáætlun}
Myndræna framsetningu má sjá á mynd \ref{fig:verkaetlun}
\subsection{Ítrun 0}
Notendasögur: 1. Rannsóknir 2. Skipulag\\
Áætlaður tími: 1+1 = 2 dagar
\subsection{Ítrun 1}
Notendasögur: 1. Sækja gögn 2. Geyma gögn í gagnagrunni\\
Áætlaður tími: 1+2 = 3 dagar

\subsection{Ítrun 2}
Notendasögur: 1. Spá fyrir um úrslit\\
Áætlaður tími: 6 dagar

\subsection{Ítrun 3}
Notendasögur: 1. Vefþjónn 2. Aðgangskerfi\\
Áætlaður tími: 4+2 = 6 dagar

\subsection{Ítrun 4}
Notendasögur: 1. Halda utan um upplýsingar um notendur 2. Nýta kerfið á öðrum vettvangi 3. Áskrift\\
Áætlaður tími: 1+2+2 = 5 dagar
\begin{figure}[H]
  \centering
  \includegraphics{verkaetlunarmynd}
  \caption{Verkáætlun}
  \label{fig:verkaetlun}
\end{figure}
\end{document}